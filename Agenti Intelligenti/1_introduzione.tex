\section{Introduzione}
Il corso ha l’obiettivo di introdurre gli aspetti principali dei sistemi multiagente, ossia sistemi composti di elementi computazionali che interagiscono, noti come agenti. Gli agenti sono sistemi computazionali capaci di eseguire azioni in modo autonomo, e di interagire con altri agenti svolgendo attività sociali come cooperazione, coordinamento, negoziazione. I sistemi multi agente costituiscono una metafora naturale per modellare un ampio spettro di “artificial social systems”. Nella prima parte del corso vengono forniti gli strumenti metodologici per comprendere i sistemi multiagente discutendo le architetture di singoli agenti e le principali problematiche legate all’interazione fra agenti. La seconda parte del corso presenta alcuni linguaggi e ambienti specifici per sistemi multiagente, in modo da consentire agli studenti di implementare alcuni esempi significativi.
\href{http://magistrale.educ.di.unito.it/index.php/offerta-formativa/insegnamenti/laurea-magistrale/insegnamenti-laurea-magistrale/scheda-insegnamento?cod=MFN1348&codA=&year=2022&orienta=YXFHO}{Pagina} del corso.