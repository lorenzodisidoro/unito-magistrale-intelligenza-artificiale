\section{Introduzione al Corso}
L'insegnamento ha l’obiettivo di approfondire le conoscenze di Intelligenza Artificiale con particolare riguardo alle capacità di un agente intelligente di fare inferenze sulla base di una rappresentazione esplicita della conoscenza sul dominio. Alle competenze metodologiche si affiancano competenze progettuali perché il corso prevede la sperimentazione di metodi di ragionamento basati sul paradigma della programmazione logica, lo sviluppo di un agente intelligente in grado di esibire sia comportamenti reattivi che deliberativi (utilizzando ambienti basati su regole di produzione) e la sperimentazione di strumenti per architetture cognitive. Continua sulla \href{http://magistrale.educ.di.unito.it/index.php/offerta-formativa/insegnamenti/laurea-magistrale/insegnamenti-laurea-magistrale/scheda-insegnamento?cod=MFN0942&codA=&year=2022&orienta=YXHO}{pagina} del corso.